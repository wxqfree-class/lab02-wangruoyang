% Created 2016-06-13 一 12:52
\documentclass[xcolor=svgnames,presentation]{beamer}
\usepackage[utf8]{inputenc}
\usepackage[T1]{fontenc}
\usepackage{fixltx2e}
\usepackage{graphicx}
\usepackage{longtable}
\usepackage{float}
\usepackage{wrapfig}
\usepackage{soul}
\usepackage{textcomp}
\usepackage{marvosym}
\usepackage{wasysym}
\usepackage{latexsym}
\usepackage{amssymb}
\usepackage{hyperref}
\tolerance=1000
\usepackage{minted}
\usecolortheme[named=FireBrick]{structure}\setbeamercovered{transparent}\setbeamertemplate{caption}[numbered]\setbeamertemplate{blocks}[rounded][shadow=true] \usetheme{Darmstadt}\date{\today} \usepackage{tikz}\usepackage{xeCJK}\usepackage{amsmath}\setmainfont{Times New Roman}\setCJKmainfont[BoldFont={Adobe Heiti Std},ItalicFont={Adobe Fangsong Std}]{Adobe Heiti Std}\setCJKsansfont{Adobe Heiti Std}\setCJKmonofont{Adobe Fangsong Std}\usepackage{verbatim}\graphicspath{{figures/}} \definecolor{lstbgcolor}{rgb}{0.9,0.9,0.9} \usepackage{listings}\usepackage{minted} \usepackage{fancyvrb}\usepackage{xcolor}\lstset{escapeinside=`',frameround=ftft,language=C,breaklines=true,keywordstyle=\color{blue!70},commentstyle=\color{red!50!green!50!blue!50},frame=shadowbox,backgroundcolor=\color{yellow!20},rulesepcolor=\color{red!20!green!20!blue!20}}
\usemintedstyle{default}
\providecommand{\alert}[1]{\textbf{#1}}

\title{第11讲 Apache服务器}
\author{王晓庆}
\date{\today}
\hypersetup{
  pdfkeywords={},
  pdfsubject={},
  pdfcreator={Emacs Org-mode version 7.9.3f}}

\institute{wangxiaoqing@outlook.com}
\begin{document}

\maketitle

\begin{frame}
\frametitle{Outline}
\setcounter{tocdepth}{1}
\tableofcontents
\end{frame}
\section{Apache服务器基本配置}
\label{sec-1}
\begin{frame}[fragile]
\frametitle{安装和管理Apache服务器}
\label{sec-1-1}
\begin{itemize}

\item 安装Apache服务器\\
\label{sec-1-1-1}%
\begin{minted}[]{bash}
yum install httpd
\end{minted}

\item 管理Apache服务器\\
\label{sec-1-1-2}%
\begin{minted}[]{bash}
chkconfig --level 345 httpd on
service httpd configtest #检查配置文件
httpd -t                 #检查配置文件
service httpd start
#客户端测试
#默认网站主目录:/var/www/html
\end{minted}
\end{itemize} % ends low level
\end{frame}
\begin{frame}
\frametitle{主配置文件/etc/httpd/conf/httpd.conf(1)}
\label{sec-1-2}
\begin{itemize}

\item 配置文件组成部分
\label{sec-1-2-1}%
\begin{enumerate}
\item 全局环境
\item 主服务器配置
\item 虚拟主机
\end{enumerate}

\item 凡是虚拟主机不能处理的请求都由主服务器处理。
\label{sec-1-2-2}%

\item 可对Apache服务器进行自上而下的分层管理:
\label{sec-1-2-3}%
\begin{enumerate}
\item 服务器(全局)
\item 网站(虚拟主机)
\item 目录(虚拟目录)
\item 文件
\end{enumerate}
\begin{itemize}

\item 下级层次的设置继承上级层次,但也可以覆盖上级设置
\label{sec-1-2-3-1}%
\end{itemize} % ends low level
\end{itemize} % ends low level
\end{frame}
\begin{frame}[fragile]
\frametitle{主配置文件/etc/httpd/conf/httpd.conf(2)}
\label{sec-1-3}
\begin{itemize}

\item 基本格式\\
\label{sec-1-3-1}%
\begin{minted}[]{bash}
指令名称 参数   #指令名称不区分大小写,但参数区分大小写
\end{minted}

\item 容器
\label{sec-1-3-2}%
\begin{itemize}

\item 用于封装一组指令,限制指令的条件或作用域\\
\label{sec-1-3-2-1}%
\begin{verbatim}
<容器名 参数>
  一组指令
</容器名>

<IfModule> 若指定模块存在则执行其内指令,否则忽略
<Directory> 目录容器
<Files> 文件容器
<Location> URL地址容器
<VirtualHost> 虚拟主机容器
\end{verbatim}
\end{itemize} % ends low level
\end{itemize} % ends low level
\end{frame}
\begin{frame}[fragile]
\frametitle{Apache服务器全局配置(1)}
\label{sec-1-4}
\begin{itemize}

\item 设置服务器根目录\\
\label{sec-1-4-1}%
\begin{minted}[]{bash}
ServerRoot "/etc/httpd" #配置、日志及相关文件的根目录
\end{minted}

\item 设置运行Apache所使用的PidFile路径\\
\label{sec-1-4-2}%
\begin{minted}[]{bash}
PidFile run/httpd.pid #记录父httpd进程的pid
\end{minted}

\item 设置连接参数\\
\label{sec-1-4-3}%
\begin{minted}[]{bash}
TimeOut #连接请求超时时间
KeepAlive #是否启用持久连接
MaxKeepAliveRequests #单个持久连接所允许的最大请求数
KeepAliveTimeout #持久连接中等待下一次请求的最长时间
\end{minted}
\end{itemize} % ends low level
\end{frame}
\begin{frame}[fragile]
\frametitle{Apache服务器全局配置(2)}
\label{sec-1-5}
\begin{itemize}

\item 配置MPM(多处理模块)(1)
\label{sec-1-5-1}%
\begin{itemize}

\item Apache是模块化的,通过MPM实现模块化功能。MPM必须静态编译,每次只能有一个MPM是活动的。目前主要使用两种模块:
\label{sec-1-5-1-1}%
\begin{enumerate}
\item 传统的prework:每个请求使用一个进程
\item 线程化的worker:使用多进程,每个进程又有多个线程
\end{enumerate}

\item 获知当前使用的MPM模型\\
\label{sec-1-5-1-2}%
\begin{minted}[]{bash}
httpd -l
\end{minted}
\end{itemize} % ends low level
\end{itemize} % ends low level
\end{frame}
\begin{frame}[fragile]
\frametitle{Apache服务器全局配置(3)}
\label{sec-1-6}
\begin{itemize}

\item 配置MPM(多处理模块)(2)
\label{sec-1-6-1}%
\begin{itemize}

\item prework默认配置\\
\label{sec-1-6-1-1}%
\begin{minted}[]{bash}
<IfModule prefork.c>
StartServers       8 #启动时的服务器进程数
MinSpareServers    5 #最小的空闲子进程数
MaxSpareServers   20 #最大的空闲子进程数
ServerLimit      256 #所允许的MaxClients最大值
MaxClients       256 #所允许的最大服务器进程数
MaxRequestsPerChild  4000 #每个进程可以处理的最大请求数
</IfModule>
\end{minted}
\end{itemize} % ends low level
\end{itemize} % ends low level
\end{frame}
\begin{frame}[fragile]
\frametitle{Apache服务器全局配置(4)}
\label{sec-1-7}
\begin{itemize}

\item 配置MPM(多处理模块)(3)
\label{sec-1-7-1}%

\item worker默认配置\\
\label{sec-1-7-2}%
\begin{minted}[]{bash}
<IfModule worker.c>
StartServers         2 #启动时的服务器进程数
MaxClients         150 #所允许的最大并发连接数
MinSpareThreads     25 #最小的空闲线程数
MaxSpareThreads     75 #最大的空闲线程数
ThreadsPerChild     25 #每个进程的线程数
MaxRequestsPerChild  0 #每个进程处理的最大请求数
</IfModule>
\end{minted}
\end{itemize} % ends low level
\end{frame}
\begin{frame}[fragile]
\frametitle{Apache服务器全局配置(5)}
\label{sec-1-8}
\begin{itemize}

\item 设置Apache服务器侦听的IP地址和端口\\
\label{sec-1-8-1}%
\begin{minted}[]{bash}
Listen [ip] 80
\end{minted}

\item 设置动态加载模块
\label{sec-1-8-2}%
\begin{itemize}

\item Apache是高度模块化的,可将模块直接编译到Apache中,也可以动态加载。动态加载模块位于/usr/lib/httpd/modules目录,可通过LoadModule语句加载。\\
\label{sec-1-8-2-1}%
\begin{minted}[]{bash}
LoadModule auth_basic_module modules/mod_auth_basic.so
LoadModule auth_digest_module modules/mod_auth_digest.so
LoadModule authn_file_module modules/mod_authn_file.so
LoadModule authn_alias_module modules/mod_authn_alias.so
LoadModule authn_anon_module modules/mod_authn_anon.so
LoadModule authn_dbm_module modules/mod_authn_dbm.so
......
\end{minted}
\end{itemize} % ends low level
\end{itemize} % ends low level
\end{frame}
\begin{frame}[fragile]
\frametitle{Apache服务器全局配置(6)}
\label{sec-1-9}
\begin{itemize}

\item 设置包含文件\\
\label{sec-1-9-1}%
\begin{minted}[]{bash}
Include conf.d/*.conf
\end{minted}

\item 设置运行Apache服务器的用户或群组\\
\label{sec-1-9-2}%
\begin{minted}[]{bash}
User apache
Group apache
\end{minted}
\end{itemize} % ends low level
\end{frame}
\begin{frame}[fragile]
\frametitle{Apache主服务器基本配置}
\label{sec-1-10}


\begin{minted}[]{bash}
ServerAdmin root@localhost #管理员email
ServerName www.example.com:80 #服务器名和端口
UseCanonicalName Off #使用客户端提供的服务器名和端口号
DocumentRoot "/var/www/html" #网站主目录(最后不能加/)
DirectoryIndex index.html index.html.var #网站默认文档
ErrorLog logs/error_log #错误日志
LogLevel warn #错误日志等级高于等于warn
CustomLog logs/access_log combined #访问日志及其格式
LogFormat "%h %l %u %t \"%r\" %>s %b \"%{Referer}i\" \
\"%{User-Agent}i\"" combined       #定义日志格式别名
LogFormat "%h %l %u %t \"%r\" %>s %b" common #同上
LogFormat "%{Referer}i -> %U" referer        #同上
LogFormat "%{User-agent}i" agent             #同上
AddDefaultCharset UTF-8             #设置默认字符集
\end{minted}
\end{frame}
\begin{frame}[fragile]
\frametitle{配置目录访问控制(1)}
\label{sec-1-11}
\begin{itemize}

\item 目录访问控制的两种实现方式
\label{sec-1-11-1}%
\begin{enumerate}
\item 在httpd.conf文件中使用<Directory>容器进行设置
\item 在目录内建立访问控制文件.htaccess,将访问控制参数写入其中,下层目录自动继承上层目录的访问控制设置
\end{enumerate}

\item 目录默认配置\\
\label{sec-1-11-2}%
\begin{minted}[]{bash}
<Directory />
    Options FollowSymLinks
    AllowOverride None
</Directory>
\end{minted}
\end{itemize} % ends low level
\end{frame}
\begin{frame}[fragile]
\frametitle{配置目录访问控制(2)}
\label{sec-1-12}
\begin{itemize}

\item 网站根目录配置\\
\label{sec-1-12-1}%
\begin{minted}[]{bash}
<Directory "/var/www/html">
    Options Indexes FollowSymLinks
    AllowOverride None #不允许被.htaccess的设置覆盖
    Order allow,deny
    Allow from all
</Directory>
\end{minted}

\item Order指令
\label{sec-1-12-2}%
\begin{itemize}

\item Order Allow,Deny\\
\label{sec-1-12-2-1}%
先匹配所有Allow指令,再匹配所有Deny指令,Deny指令可以覆盖Allow指令,若两者均无匹配则拒绝访问。

\item Order Deny,Allow\\
\label{sec-1-12-2-2}%
先匹配所有Deny指令,再匹配所有Allow指令,Allow指令可以覆盖Deny指令,若两者均无匹配则允许访问。

\item Order Mutual-failure\\
\label{sec-1-12-2-3}%
效果与Order Allow,Deny相同,但不建议使用。
\end{itemize} % ends low level
\end{itemize} % ends low level
\end{frame}
\begin{frame}[fragile]
\frametitle{配置目录访问控制(3)}
\label{sec-1-13}
\begin{exampleblock}{示例1}
\label{sec-1-13-1}


\begin{minted}[]{bash}
Order Deny,Allow
Deny from all
Allow from abc.com
#拒绝所有客户端访问,但允许abc.com域的主机访问
\end{minted}
\end{exampleblock}
\begin{block}{示例2}
\label{sec-1-13-2}


\begin{minted}[]{bash}
Order Allow,Deny
Allow from xyz.net
Deny from jx.xyz.net
#允许来自xyz.net域的主机访问(jx.xyz.net子域除外)
#拒绝所有其他主机
\end{minted}
\end{block}
\begin{itemize}

\item 若示例2改为Order Deny,Allow,则哪些主机可以访问?
\label{sec-1-13-3}%
\end{itemize} % ends low level
\note{答案

所有主机均可以访问!
}
\end{frame}
\begin{frame}[fragile]
\frametitle{配置和管理虚拟目录}
\label{sec-1-14}
\begin{itemize}

\item 虚拟目录和物理目录
\label{sec-1-14-1}%
\begin{itemize}

\item 物理目录:网站主目录内的实际子目录
\label{sec-1-14-1-1}%

\item 虚拟目录:网站主目录之外的其他目录,但在URL中显示为网站子目录,方便灵活
\label{sec-1-14-1-2}%

\item 物理目录与虚拟目录重名时,优先访问虚拟目录。
\label{sec-1-14-1-3}%

\item ISP常使用虚拟目录提供免费个人主页,企业往往使用虚拟目录作为各部门的子网站。
\label{sec-1-14-1-4}%
\end{itemize} % ends low level

\item 创建虚拟目录\\
\label{sec-1-14-2}%
\begin{minted}[]{bash}
Alias /icons/ "/var/www/icons/"
Alias /error/ "/var/www/error/"
\end{minted}

\item 根据需要还可以对相应的实际物理目录设置访问权限。
\label{sec-1-14-3}%
\end{itemize} % ends low level
\end{frame}
\begin{frame}[fragile]
\frametitle{为用户配置个人Web空间(1)}
\label{sec-1-15}
\begin{itemize}

\item 1. 修改/etc/httpd/conf/httpd.conf文件\\
\label{sec-1-15-1}%
\begin{minted}[]{bash}
<IfModule mod_userdir.c>
    UserDir disable root #禁止root用户使用个人主页
    UserDir public_html  #设置用户Web站点目录
</IfModule>
#取消注释以下容器及其指令
<Directory /home/*/public_html>
    AllowOverride FileInfo AuthConfig Limit
    ......
</Directory>
\end{minted}
\end{itemize} % ends low level
\end{frame}
\begin{frame}[fragile]
\frametitle{为用户配置个人Web空间(2)}
\label{sec-1-16}
\begin{itemize}

\item 2. 创建用户个人主页主目录并修改权限\\
\label{sec-1-16-1}%
\begin{minted}[]{bash}
su - mike
mkdir public_html
chmod 711 ~mike
chmod 755 ~mike/public_html
\end{minted}

\item 3. 在public$_{\mathrm{html}}$目录中创建index.html文件
\label{sec-1-16-2}%

\item 4. 重启httpd服务,或重新加载httpd配置文件\\
\label{sec-1-16-3}%
\begin{minted}[]{bash}
service httpd restart|reload
\end{minted}

\item 5. 测试访问\\
\label{sec-1-16-4}%
\begin{minted}[]{bash}
http://192.168.0.200/~mike
\end{minted}
\end{itemize} % ends low level
\end{frame}
\section{配置Web应用程序}
\label{sec-2}
\begin{frame}[fragile]
\frametitle{配置PHP应用程序}
\label{sec-2-1}
\begin{itemize}

\item 1. 安装PHP解释器\\
\label{sec-2-1-1}%
\begin{minted}[]{bash}
yum install php-common php-cli php
\end{minted}

\item 2. 配置Apache以支持PHP\\
\label{sec-2-1-2}%
\begin{minted}[]{bash}
vim /etc/httpd/conf/httpd.conf
...
Include conf.d/*.conf

cat /etc/httpd/conf.d/php.conf
ls /etc/php.ini #php本身的配置文件
\end{minted}

\item 3. 测试\\
\label{sec-2-1-3}%
\begin{minted}[]{bash}
cd /var/www/html
cat '<?phpinfo()?>' >test.php #创建php测试页面
访问http://192.168.56.200/test.php
\end{minted}
\end{itemize} % ends low level
\end{frame}
\begin{frame}[fragile]
\frametitle{配置和管理MySQL数据库服务器(1)}
\label{sec-2-2}
\begin{itemize}

\item 1. 安装MySQL及相关程序\\
\label{sec-2-2-1}%
\begin{minted}[]{bash}
yum install perl-DBI perl-DBD-MySQL mysql mysql-server
service mysqld start
#管理员账户为root,默认密码为空
mysqladmin -u root password 123456 #首次设置root用户密码
mysqladmin -u root -p password 111111 #修改root用户密码
mysql -uroot -p111111 #登录mysql,-p后面不能有空格!
mysql> show databases; #后面要加分号
mysql> exit
\end{minted}
\end{itemize} % ends low level
\end{frame}
\begin{frame}[fragile]
\frametitle{配置和管理MySQL数据库服务器(2)}
\label{sec-2-3}
\begin{itemize}

\item 2. 使用phpMyAdmin管理MySQL\\
\label{sec-2-3-1}%
\begin{minted}[]{bash}
yum install php-mysql php-mbstring
#下载phpMyAdmin至/var/www/html目录
cd /var/www/html
tar -xjvf phpMyAdmin-2.11.11.3-all-languages.tar.bz2
mv phpMyAdmin-2.11.11.3-all-languages phpMyAdmin
cd phpMyAdmin
cp -p config.sample.inc.php config.inc.php
vim config.inc.php
$cfg['blowfish_secret'] = 'www.abc.com';
...
$cfg['Servers'][$i]['user'] = 'root';
$cfg['Servers'][$i]['password'] = '111111';
访问http://192.168.56.200/phpMyAdmin
\end{minted}
\end{itemize} % ends low level
\end{frame}
\section{配置和管理虚拟主机}
\label{sec-3}
\begin{frame}[fragile]
\frametitle{基于IP的虚拟主机(1)}
\label{sec-3-1}


\begin{minted}[]{bash}
#1. 为服务器配置多块网卡或配置虚拟网卡
ifconfig eth0:0 192.168.56.201
ifconfig eth0:1 192.168.56.202
#2. 为虚拟主机注册域名,此处为方便可直接修改hosts文件
vim /etc/hosts
......
192.168.56.201  www.abc.com
192.168.56.202  www.xyz.net
#3. 为两个网站分别创建网站根目录
mkdir -p /var/www/abc.com
mkdir -p /var/www/xyz.net
#4. 在两个网站的根目录中分别创建index.html文件
#5. 编辑httpd.conf文件,确认配置有以下Listen指令
Listen 80
\end{minted}
\end{frame}
\begin{frame}[fragile]
\frametitle{基于IP的虚拟主机(2)}
\label{sec-3-2}


\begin{minted}[]{bash}
#6. 编辑httpd.conf文件,定义虚拟主机
<VirtualHost 192.168.56.201>
    ServerName www.abc.com
    DocumentRoot /var/www/abc.com
</VirtualHost>
<VirtualHost 192.168.56.202>
    ServerName www.xyz.net
    DocumentRoot /var/www/xyz.net
</VirtualHost>
#7. 重启httpd服务
service httpd restart
#8. 测试访问
1. 如果配置了DNS,则即可以通过ip也可以通过域名访问。
2. 对于<VritualHost>容器中未定义地址的请求(如localhost),
   都将指向主服务器。
\end{minted}
\end{frame}
\begin{frame}
\frametitle{基于名称的虚拟主机(1)}
\label{sec-3-3}
\begin{itemize}

\item 定义
\label{sec-3-3-1}%
\begin{itemize}

\item 将多个域名绑定到同一个IP地址,服务器通过HTTP请求中的主机名来确定客户请求的是哪个网站。
\label{sec-3-3-1-1}%
\end{itemize} % ends low level

\item 特点
\label{sec-3-3-2}%
\begin{itemize}

\item 节省IP地址,只能通过域名访问,不能通过IP地址访问
\label{sec-3-3-2-1}%
\end{itemize} % ends low level

\item 虚拟主机匹配顺序
\label{sec-3-3-3}%
\begin{itemize}

\item 检查HTTP请求是否使用了与NameVirtualHost指令匹配的ip地址,如果是,则逐一查找使用该IP地址的<VirtualHost>段,并尝试找出一个ServerName或ServerAlias指令与该请求匹配的虚拟主机。如果无匹配,则使用符合该地址的第一个虚拟主机,即排在最前面的虚拟主机成为默认虚拟主机。
\label{sec-3-3-3-1}%

\item 当请求的IP地址与NameVirtualHost指令中的地址匹配时,主服务器中的DocumentRoot将永远不会被用到。
\label{sec-3-3-3-2}%
\end{itemize} % ends low level
\end{itemize} % ends low level
\end{frame}
\begin{frame}[fragile]
\frametitle{基于名称的虚拟主机(2)}
\label{sec-3-4}
\begin{itemize}

\item 示例1:在单一IP地址上运行多个基于名称的网站(1)\\
\label{sec-3-4-1}%
\begin{minted}[]{bash}
#1. 配置域名解析,为方便可以编辑hosts文件
vim /etc/hosts
192.168.56.200  www.abc.com
192.168.56.200  www.xyz.net
#2. 为每个网站创建根目录
mkdir -p /var/www/abc.com
mkdir -p /var/www/xyz.net
#3. 在每个网站根目录下准备index.html文件
#4. 编辑httpd.conf文件
......
Listen 80
......
\end{minted}
\end{itemize} % ends low level
\end{frame}
\begin{frame}[fragile]
\frametitle{基于名称的虚拟主机(3)}
\label{sec-3-5}
\begin{itemize}

\item 示例1:在单一IP地址上运行多个基于名称的网站(2)\\
\label{sec-3-5-1}%
\begin{minted}[]{bash}
#5. 配置虚拟主机
NameVirtualHost *:80 #侦听所有IP地址的虚拟主机请求
<VirtualHost *:80>   #地址要与NameVirtualHost指令一致
    ServerName www.abc.com
    DocumentRoot /var/www/abc.com
</VirtualHost>
<VirtualHost *:80>
    ServerName www.xyz.net
    DocumentRoot /var/www/xyz.net
</VirtualHost>
#6. 重启httpd服务并测试
service httpd restart
\end{minted}

\item 说明:本例中*与任何地址均匹配,因此www.abc.com将成为默认或主服务器。
\label{sec-3-5-2}%
\end{itemize} % ends low level
\end{frame}
\begin{frame}[fragile]
\frametitle{基于名称的虚拟主机(4)}
\label{sec-3-6}
\begin{itemize}

\item 示例2:在多个IP地址上运行基于名称的Web网站(2)\\
\label{sec-3-6-1}%
\begin{minted}[]{bash}
Listen 80
ServerName www.mainsite.com
DocumentRoot "/var/www/html"
NameVirtualHost 192.168.56.201
<VirtualHost 192.168.56.201>
    ServerName www.abc.com
    DocumentRoot /var/www/abc.com
</VirtualHost>
<VirtualHost 192.168.56.201>
    ServerName www.xyz.net
    DocumentRoot /var/www/xyz.net
</VirtualHost>
\end{minted}

\item 说明:对于192.168.56.201之外的访问均由主服务器响应,而www.abc.com则是192.168.56.201上的默认服务器。
\label{sec-3-6-2}%
\end{itemize} % ends low level
\end{frame}
\begin{frame}[fragile]
\frametitle{基于名称的虚拟主机(5)}
\label{sec-3-7}
\begin{itemize}

\item 示例3:在不同地址(内外网)上运行相同的网站\\
\label{sec-3-7-1}%
\begin{minted}[]{bash}
NameVirtualHost 192.168.56.200
NameVirtualHost 200.1.1.200
<VirtualHost 192.168.56.200 200.1.1.200>
    ServerName www.abc.com
    ServerAlias webserver
    DocumentRoot /var/www/abc.com
</VirtualHost>
\end{minted}

\item 内网用户可以使用ServerAlias定义的别名webserver,而不用www.abc.com来访问网站。
\label{sec-3-7-2}%
\end{itemize} % ends low level
\end{frame}
\begin{frame}[fragile]
\frametitle{基于TCP端口号架设多个Web网站}
\label{sec-3-8}
\begin{itemize}

\item 示例\\
\label{sec-3-8-1}%
\begin{minted}[]{bash}
Listen 80
Listen 8080
<VirtualHost 192.168.56.200:80>
    ServerName www.abc.com
    DocumentRoot /var/www/abc.com
</VirtualHost>
<VirtualHost 192.168.56.200:8080>
    ServerName www.xyz.net
    DocumentRoot /var/www/xyz.net
</VirtualHost>
\end{minted}
\end{itemize} % ends low level
\end{frame}
\section{配置Web服务器安全}
\label{sec-4}
\begin{frame}[fragile]
\frametitle{用户认证(1)}
\label{sec-4-1}
\begin{itemize}

\item 认证指令:可以出现在<Directory>容器或.htaccess文件中。\\
\label{sec-4-1-1}%
\begin{minted}[]{bash}
AuthType Basic|Digest
#Digest更安全,但并非所有浏览器都支持,因此通常用Basic
AuthName "Please login: "        #设置登录提示内容
AuthUserFile /etc/httpd/testpwd  #设置用户密码文件
AuthGroupFile /etc/httpd/testgrp #设置组密码文件
\end{minted}

\item 授权命令:为指定用户/组授权访问资源\\
\label{sec-4-1-2}%
\begin{minted}[]{bash}
Require user usr1 usr2 ...  #为指定用户授权
Require group grp1 grp2 ... #为指定组授权
Require valid-user #为认证密码文件中所有用户授权
\end{minted}
\end{itemize} % ends low level
\end{frame}
\begin{frame}[fragile]
\frametitle{用户认证(2)}
\label{sec-4-2}
\begin{itemize}

\item 管理密码文件\\
\label{sec-4-2-1}%
\begin{minted}[]{bash}
用户名:加密的密码 #密码文件格式
htpasswd [-c][-m][-D] 密码文件名 用户名
#-c:创建密码文件,若文件已存在则清空并改写
#-m:使用md5加密密码
#-D:如果用户存在于密码文件中,则删除该用户
htpasswd -c file user #添加认证用户并创建密码文件
htpasswd file user #在现有密码文件中添加/修改用户密码
\end{minted}
\end{itemize} % ends low level
\begin{block}{注意}
\label{sec-4-2-2}

密码文件应存储在不能被网络用户访问的位置,以免被下载!
\end{block}
\end{frame}
\begin{frame}[fragile]
\frametitle{用户认证(3)}
\label{sec-4-3}
\begin{itemize}

\item 示例:使用基本认证方法实现Web用户认证\\
\label{sec-4-3-1}%
\begin{minted}[]{bash}
#1. 为用户mike创建一个密码文件
htpasswd -c /etc/httpd/passwords mike
#2. 配置Web服务器,要求经过认证才能访问某网站/目录
<Directory "/var/www/html/dev">
    AuthType Basic
    AuthName "Restricted Files:"
    AuthUserFile /etc/httpd/passwds
    Require user mike #授权mike可以访问该目录
</Directory>
#3. 重启httpd服务
service httpd restart
#4. 访问测试
http://192.168.56.200/dev
\end{minted}
\end{itemize} % ends low level
\end{frame}
\begin{frame}[fragile]
\frametitle{访问控制}
\label{sec-4-4}
\begin{itemize}

\item 限制目录访问:<Directory>容器
\label{sec-4-4-1}%

\item 限制文件访问:<Files>容器\\
\label{sec-4-4-2}%
\begin{minted}[]{bash}
<Files ~ "^\.ht"> #文件名以.ht开头的文件
    Order allow,deny
    Deny from all #拒绝所有客户端访问
</Files>
\end{minted}

\item 限制URL地址访问:<Location>容器\\
\label{sec-4-4-3}%
\begin{minted}[]{bash}
<Location /inner> #网站中以/inner开头的URL
    Order deny,allow
    Deny from all
    Allow from 192.168.0.1
</Location>
\end{minted}

\item 通过本地文件权限控制访问
\label{sec-4-4-4}%
\end{itemize} % ends low level
\end{frame}
\begin{frame}
\frametitle{为Apache服务器配置SSL(1)}
\label{sec-4-5}
\begin{itemize}

\item 基于SSL的Web网站可以实现以下安全目标:
\label{sec-4-5-1}%
\begin{enumerate}
\item Web浏览器确认Web服务器的身份,防止假冒网站。
\item 在Web服务器和Web浏览器之间建立安全的数据通道,确保安全传输敏感数据,防止数据被第三方非法获取。
\item 如有必要,可以让服务器确认用户身份,防止假冒用户。
\end{enumerate}

\item 假设SSL安全网站,关键要具备以下条件:
\label{sec-4-5-2}%
\begin{enumerate}
\item 需要从可信的或权威的证书颁发机构(CA)获取证书,也可以创建自签名的证书(X509结构)。另外还要保证证书不能过期。
\item 必须在Web服务器上安装服务器证书并启用SSL功能。
\item 如果要求对客户端进行身份验证,则客户端需要申请和安装用户证书;否则,客户端必须与服务器信任同一证书认证机构,需要安装CA证书。
\end{enumerate}
\end{itemize} % ends low level
\end{frame}
\begin{frame}[fragile]
\frametitle{为Apache服务器配置SSL(2)}
\label{sec-4-6}
\begin{itemize}

\item 1. 安装必要的软件包\\
\label{sec-4-6-1}%
\begin{minted}[]{bash}
rpm -qa | grep openssl
rpm -qa | grep mod_ssl
\end{minted}

\item 2. 为Apache服务器准备SSL证书\\
\label{sec-4-6-2}%
\begin{minted}[]{bash}
cd /etc/pki/tls/private
openssl genrsa -out abcsrv.key 1024 #为服务器创建私钥
#利用上一步产生的私钥创建一个证书签名请求文件
openssl req -new -key abcsrv.key -out abcsrv.csr
mkdir /etc/pki/tls/csr; mv abcsrv.csr /etc/pki/tls/csr
cd /etc/pki/tls/certs
#基于服务器私钥为服务器创建一个自签名证书
openssl x509 -req -days 365 -in \
/etc/pki/tls/csr/abcsrv.csr -signkey \
/etc/pki/tls/private/abcsrv.key -out abcsrv.crt
\end{minted}
\end{itemize} % ends low level
\end{frame}
\begin{frame}[fragile]
\frametitle{为Apache服务器配置SSL(3)}
\label{sec-4-7}
\begin{itemize}

\item 3. 为Apache服务器启用SSL功能\\
\label{sec-4-7-1}%
\begin{minted}[]{bash}
vim /etc/httpd/conf.d/ssl.conf
LoadModule ssl_module modules/mod_ssl.so
Listen 443
<VirtualHost _default_:443>
SSLEngine on  #启用SSL功能
#设置服务器证书文件
SSLCertificateFile /etc/pki/tls/certs/abcsrv.crt
#设置服务器私钥文件
SSLCertificateKeyFile /etc/pki/tls/private/abcsrv.key
</VirtualHost>
\end{minted}

\item 4. 重启httpd服务\\
\label{sec-4-7-2}%
\begin{minted}[]{bash}
service httpd restart
\end{minted}
\end{itemize} % ends low level
\end{frame}
\begin{frame}[fragile]
\frametitle{为Apache服务器配置SSL(4)}
\label{sec-4-8}
\begin{itemize}

\item 5. 客户端基于SSL连接到Apache服务器\\
\label{sec-4-8-1}%
\begin{minted}[]{bash}
访问 https://www.abc.com
\end{minted}

\item 强制客户端使用SSL连接
\label{sec-4-8-2}%
\begin{itemize}

\item 按照上述配置,HTTP和HTTPS两种通信连接都支持。如果要强制客户端使用HTTPS,只要屏蔽非SSL网站即可。例如,不允许侦听80端口,或不配置80端口的虚拟主机。
\label{sec-4-8-2-1}%
\end{itemize} % ends low level
\end{itemize} % ends low level
\end{frame}
\begin{frame}[fragile]
\frametitle{为Apache服务器配置SSL(5)}
\label{sec-4-9}
\begin{itemize}

\item 为Apache虚拟主机启用SSL(1)
\label{sec-4-9-1}%
\begin{itemize}

\item 基于IP的虚拟主机:需为每个域名/IP申请证书\\
\label{sec-4-9-1-1}%
\begin{minted}[]{bash}
Listen 443
<VirtualHost 192.168.56.201:443>
SSLEngine on
SSLCertificateFile /etc/pki/tls/certs/abcsrv.crt
SSLCertificateKeyFile /etc/pki/tls/private/abcsrv.key
......
</VirtualHost>
<VirtualHost 192.168.56.202:443>
SSLEngine on
SSLCertificateFile /etc/pki/tls/certs/xyzsrv.crt
SSLCertificateKeyFile /etc/pki/tls/private/xyzsrv.key
......
</VirtualHost>
\end{minted}
\end{itemize} % ends low level
\end{itemize} % ends low level
\end{frame}
\begin{frame}[fragile]
\frametitle{为Apache服务器配置SSL(6)}
\label{sec-4-10}
\begin{itemize}

\item 为Apache虚拟主机启用SSL(2)
\label{sec-4-10-1}%
\begin{itemize}

\item 基于TCP端口的虚拟主机:只需为一个域名/IP申请证书\\
\label{sec-4-10-1-1}%
\begin{minted}[]{bash}
Listen 443
Listen 8443
<VirtualHost 192.168.56.201:443>
SSLEngine on
SSLCertificateFile /etc/pki/tls/certs/abcsrv.crt
SSLCertificateKeyFile /etc/pki/tls/private/abcsrv.key
......
</VirtualHost>
<VirtualHost 192.168.56.201:8443>
SSLEngine on
SSLCertificateFile /etc/pki/tls/certs/abcsrv.crt
SSLCertificateKeyFile /etc/pki/tls/private/abcsrv.key
......
</VirtualHost>
\end{minted}
\end{itemize} % ends low level
\end{itemize} % ends low level
\end{frame}
\section{管理Apache服务器}
\label{sec-5}
\begin{frame}[fragile]
\frametitle{监控Apache服务器}
\label{sec-5-1}
\begin{itemize}

\item 编辑httpd.conf,去掉以下指令前的注释:\\
\label{sec-5-1-1}%
\begin{minted}[]{bash}
<Location /server-status>
    SetHandler server-status
    Order deny,allow
    Deny from all
    Allow from .example.com
</Location>
\end{minted}

\item 访问 \href{http://192.168.56.200/server-status}{http://192.168.56.200/server-status}
\label{sec-5-1-2}%
\end{itemize} % ends low level
\end{frame}
\begin{frame}[fragile]
\frametitle{查看Apache服务器配置信息}
\label{sec-5-2}
\begin{itemize}

\item 编辑httpd.conf,去掉以下指令前的注释:\\
\label{sec-5-2-1}%
\begin{minted}[]{bash}
<Location /server-info>
    SetHandler server-status
    Order deny,allow
    Deny from all
    Allow from .example.com
</Location>
\end{minted}

\item 访问 \href{http://192.168.56.200/server-info}{http://192.168.56.200/server-info}
\label{sec-5-2-2}%
\end{itemize} % ends low level
\end{frame}
\begin{frame}[fragile]
\frametitle{查看和分析Apache服务器日志(1)}
\label{sec-5-3}
\begin{itemize}

\item 检查错误日志和访问日志\\
\label{sec-5-3-1}%
\begin{minted}[]{bash}
vim /etc/httpd/conf/httpd.conf
......
ErrorLog logs/error_log           #配置错误日志
......
CustomLog logs/access_log common  #配置访问日志

tail -f /etc/httpd/logs/error_log #实时监测错误日志
\end{minted}
\end{itemize} % ends low level
\end{frame}
\begin{frame}[fragile]
\frametitle{查看和分析Apache服务器日志(2)}
\label{sec-5-4}
\begin{itemize}

\item 使用Webalizer分析访问日志\\
\label{sec-5-4-1}%
\begin{minted}[]{bash}
vim /etc/httpd/conf.d/webalizer.conf
Alias /usage /var/www/usage
<Location /usage>
    Order deny,allow
    Deny from all
    Allow from 127.0.0.1
    Allow from ::1
    Allow from 192.168.56.1 #增加客户端许可
</Location>

service httpd restart
访问 http://192.168.56.200/usage
\end{minted}
\end{itemize} % ends low level
\end{frame}

\end{document}
