% Created 2016-02-22 一 12:08
\documentclass[bigger]{beamer}
\usepackage[utf8]{inputenc}
\usepackage[T1]{fontenc}
\usepackage{fixltx2e}
\usepackage{graphicx}
\usepackage{longtable}
\usepackage{float}
\usepackage{wrapfig}
\usepackage{soul}
\usepackage{textcomp}
\usepackage{marvosym}
\usepackage{wasysym}
\usepackage{latexsym}
\usepackage{amssymb}
\usepackage{hyperref}
\tolerance=1000
\usepackage{xeCJK}
\providecommand{\alert}[1]{\textbf{#1}}

\title{Linux操作系统}
\author{wxq}
\date{2016-02-22 一}
\hypersetup{
  pdfkeywords={},
  pdfsubject={},
  pdfcreator={Emacs Org-mode version 7.9.3f}}

\begin{document}

\maketitle

\begin{frame}
\frametitle{Outline}
\setcounter{tocdepth}{3}
\tableofcontents
\end{frame}

\section{第1周}
\label{sec-1}
\begin{frame}[fragile]
\frametitle{1. 课程简介}
\label{sec-1-1}
\begin{itemize}

\item 1.1 为什么要学Linux?
\label{sec-1-1-1}%
\begin{itemize}
\item Linux与IT行业
\begin{enumerate}
\item 免费、开源、全球参与开发
\item 被业界广泛采用,成为全球IT基础设施(尤其是因特网)核心
\item 在移动和物联网时代发展愈加迅猛
\item 是各种新技术产生的温床
\end{enumerate}
\item Linux与我们
\begin{enumerate}
\item 可是我们平时都只用Windows,从来没用过Linux呀?
\begin{itemize}
\item 不上网:Windows
\item 上网:新浪网易看新闻、天猫京东购物、网上购买回家火车票、QQ微信、收发邮件、炒股、玩手机、看有线电视、看视觉大片
\end{itemize}
\item 与Unix一脉相承,承载了丰富而历久弥新的知识和软件财富
\item 通过学习命令行和脚本编程,可以实现高效且自动化地处理各种任务
\item 通过学习系统配置管理,可以深入掌控系统的方方面面
\item 学到的知识可以保值并随着积累不断增值
\item 可以第一时间接触和了解业界最新技术
\end{enumerate}
\end{itemize}

\item 1.2 主要学什么?
\label{sec-1-1-2}%
\begin{itemize}
\item shell命令行
\begin{enumerate}
\item GUI(图形用户界面)
\begin{itemize}
\item 拟物化:人操作物
\begin{itemize}
\item 例:双击图标运行程序(人亲自执行)
\end{itemize}
\item 适合视觉型活动:图形图像设计、音视频制作、上网、看视频、玩游戏
\item 面向内容消费:多媒体相关工作者、普通大众
\item 例:制作网页logo图片
\end{itemize}
\item CLI(命令行界面)
\begin{itemize}
\item 拟人化:人机对话
\begin{itemize}
\item 例:命令机器执行程序(让机器执行)
\end{itemize}
\item 适合语言型活动:编程、写文档、记笔记
\item 面向内容生产:计算机专业人士
\item 例:
\begin{itemize}
\item 创建3个文件夹memo、book、note,memo文件夹内创建12个文件夹(用于分别存放12个月的备忘)、book文件夹内创建3个文件夹(cs、math、en)、note文件夹内创建3个文件夹(os、net、db),每个文件夹内再创建16个子文件夹(用于分别存放每周的课程笔记)

\begin{verbatim}
mkdir -p memo/month{01..12} book/{cs,math,en} note/{os,net,db}/week{01..16}
\end{verbatim}
\end{itemize}
\end{itemize}
\end{enumerate}
\item shell脚本编程
\item 系统配置管理
\item 网络服务配置管理
\end{itemize}

\item 1.3 怎么考核?
\label{sec-1-1-3}%
\begin{itemize}
\item 考勤:20\%
\begin{itemize}
\item 随机考勤,每次旷课扣3分(早退按旷课处理),每次迟到扣1分,扣完为止。
\end{itemize}
\item 实践:30\%
\begin{enumerate}
\item 平时:14\%
\begin{itemize}
\item 选项1:根据每周上机实验情况打分
\item 选项2:根据完成课程项目情况打分
\item 说明:为在规定时间完成实验或上交课程项目,分数乘0.7进行折扣
\end{itemize}
\item 期末:16\%
\begin{itemize}
\item 未在规定时间内完成,分数乘0.8进行折扣
\end{itemize}
\end{enumerate}
\item 期末笔试:50\%
\begin{itemize}
\item 闭卷考试,卷面分数不得低于50分,否则总评分会被系统自动定为不及格!
\end{itemize}
\end{itemize}
\end{itemize} % ends low level
\end{frame}
\begin{frame}
\frametitle{2. 从Unix到Linux}
\label{sec-1-2}
\begin{itemize}

\item 2.1 Unix简史
\label{sec-1-2-1}%

\item 2.2 自由软件和开源软件
\label{sec-1-2-2}%

\item 2.3 Linux简介
\label{sec-1-2-3}%

\item 2.4 Linux vs. Windows
\label{sec-1-2-4}%
\end{itemize} % ends low level
\end{frame}
\begin{frame}
\frametitle{3. Linux安装}
\label{sec-1-3}
\begin{itemize}

\item 3.1 通过虚拟机安装Linux
\label{sec-1-3-1}%

\item 3.2 Linux分区
\label{sec-1-3-2}%
\begin{itemize}
\item 分区与目录
\begin{enumerate}
\item 硬盘分区规则
\item Linux分区与Windos分区
\begin{itemize}
\item 单根目录与多根目录
\end{itemize}
\item Linux分区与目录(挂载点)
\end{enumerate}
\item 分区方案
\begin{enumerate}
\item /+Swap
\item /+/boot+Swap
\item /+/boot+/home+Swap
\item \ldots{}...
\end{enumerate}
\item 分区名称
\begin{enumerate}
\item IDE接口设备及其分区
\item SCSI接口设备及其分区
\item 设备文件名 vs. 挂载点
\end{enumerate}
\end{itemize}

\item 3.3 虚拟机网络设置
\label{sec-1-3-3}%
\begin{itemize}
\item NAT
\item Bridged
\item Host-only
\item Internel
\end{itemize}

\item 3.4 通过putty远程登录Linux
\label{sec-1-3-4}%
\begin{itemize}
\item 设置NAT端口转发
\end{itemize}

\item 3.5 虚拟机管理
\label{sec-1-3-5}%
\begin{itemize}
\item 虚拟机快照保存与恢复
\item 利用虚拟硬盘快速创建虚拟机
\item 虚拟机复制
\end{itemize}
\end{itemize} % ends low level
\end{frame}
\begin{frame}[fragile]
\frametitle{4. Linux入门}
\label{sec-1-4}
\begin{itemize}

\item 4.1 终端与多用户
\label{sec-1-4-1}%
\begin{itemize}
\item 本地终端
\begin{enumerate}
\item 字符终端
\item 图形终端
\end{enumerate}
\item 远程终端
\end{itemize}

\item 4.2 入门使用
\label{sec-1-4-2}%
\begin{itemize}
\item 登录
\item 了解系统
\begin{itemize}
\item 时间

\begin{verbatim}
date    #现在什么时间?
date +%H:%M  #现在几点几分?
date "+%B %d"  #今天是几月几号呀?
date +%s     #打印纪元时(秒):Linux认为UTC 1970年1月1日0点整是纪元时间
echo `date +%s`/265/24/3600 | bc  #现在是纪元多少年呀?
date --date "Oct 1 2016" +%A      #今年国庆节是星期几呀?
date -s "10 March 2016 10:01:23"  #帮我把时间调成2016年3月10日10点01分23秒吧!
\end{verbatim}
\item 日历

\begin{verbatim}
cal #给我看看这个月的月历吧
cal 2017 #我想看看2017年的年历
cal 9 1752  #我想看看1752年9月份的月历:-)
\end{verbatim}
\item 用户名

\begin{verbatim}
whoami  #我是谁?
who am i #我究竟是谁?
who     #都有谁在呀?
\end{verbatim}
\item 系统版本

\begin{verbatim}
uname  #这是什么操作系统呀?
uname -r #这是什么版本呀?
uname -a #把操作系统相关信息都告诉我好了!
\end{verbatim}
\item 主机名

\begin{verbatim}
hostname #主机名是什么呀?
hostname NiuBi #不好听,改个牛逼的名字吧!
\end{verbatim}
\item 系统状态

\begin{verbatim}
uptime  #开机多久了呀?
uptime -p  #拜托,友好点行吗?
\end{verbatim}
\item 计算器

\begin{verbatim}
bc    #我想算点东西
bc -q #我只想安静地算点东西!
echo "scale=16;x=4;y=7;3*(x+2)/y" | bc #帮我算一下这个吧!
echo "obase=2;192" | bc  #192的二进制表示是什么呀?
echo "obase=10;ibase=2;11000011" | bc #11000011代表几呀?
echo "ibase=2;obase=10;11000011" | bc #:-(
echo "ibase=2;obase=1010;11000011" | bc #:-)
\end{verbatim}
\end{itemize}
\item 浏览文件系统
\begin{enumerate}
\item ls
\item cd
\item pwd
\item cat
\item Linux目录架构(与Windows对比)
\item 绝对路径和相对路径
\end{enumerate}
\item 切换终端
\item 切换用户身份
\item 与其他用户交互
\begin{enumerate}
\item who
\item write和mesg
\item mail
\end{enumerate}
\item 注销
\item 关机
\end{itemize}

\item 4.3 命令行
\label{sec-1-4-3}%
\begin{itemize}
\item 命令格式:命令 [选项]\ldots{} [参数]\ldots{}
\begin{enumerate}
\item 选项(指示命令以什么方式执行)
\begin{itemize}
\item Unix简洁风:-a
\begin{itemize}
\item 选项a-z巡礼

\begin{center}
\begin{tabular}{ll}
 选项    &  含义                                    \\
\hline
 -a      &  all, append                             \\
 -b      &  buffer, block, batch                    \\
 -c      &  command, check                          \\
 -d      &  debug, delete, directory                \\
 -D      &  define                                  \\
 -e      &  excute, edit, exclude, expression       \\
 -f      &  file, force                             \\
 -h      &  header, help                            \\
 -i      &  initialize, interactive                 \\
 -I      &  include                                 \\
 -k      &  keep, kill                              \\
 -l      &  list, long, load, login                 \\
 -m      &  message, mail, mode, modification-time  \\
 -n      &  number, not                             \\
 -o      &  output                                  \\
 -p      &  port, protocol                          \\
 -q      &  quite                                   \\
 -r(-R)  &  recurse, reverse                        \\
 -s      &  silent, subject, size                   \\
 -t      &  tag                                     \\
 -u      &  user                                    \\
 -v      &  verbose, version                        \\
 -V      &  version                                 \\
 -w      &  width, warning                          \\
 -x      &  debug, extract                          \\
 -y      &  yes                                     \\
 -z      &  zip                                     \\
\end{tabular}
\end{center}


\end{itemize}
\item GNU友好风:--all
\end{itemize}
\item 参数(指示命令作用的对象,选项也可以有参数)
\item 注意:命令、选项和参数之间要有空格(选项及其参数之间有时可以没有空格)
\end{enumerate}
\item 命令编辑:Tab键自动完成
\item 命令历史:history
\item 中止命令:Ctrl+c
\end{itemize}

\item 4.4 获取帮助
\label{sec-1-4-4}%
\begin{itemize}
\item 外部命令与内部命令
\item man
\item help
\item info
\item doc
\item Internet
\end{itemize}
\end{itemize} % ends low level
\end{frame}
\section{第2周}
\label{sec-2}
\begin{frame}[fragile]
\frametitle{1. Linux文件管理}
\label{sec-2-1}

\begin{itemize}
\item 文件
\begin{enumerate}
\item 创建文件
\begin{itemize}
\item 文件名规范
\item touch

\begin{verbatim}
touch file1
\end{verbatim}
\end{itemize}
\item 复制文件q
\begin{itemize}
\item cp

\begin{verbatim}
cp file1 file2  #把file1复制一份并命名为file2
cp file file2 dir1  #把file1、file2复制到目录dir1内
cp file dir1/file3  #把file1复制到目录dir1内并命名为file3
\end{verbatim}
\end{itemize}
\item 移动文件
\begin{itemize}
\item mv

\begin{verbatim}
mv file1 file2 #将file1移动为file2(即重命名)
mv file2 file3 file4 dir1  #将file2、file3、file4移动到目录dir1内
mv file5 dir1/file1 #将file5移动到目录dir1内,并重命名为file1
\end{verbatim}
\end{itemize}
\item 删除文件
\begin{itemize}
\item rm

\begin{verbatim}
rm file1 file2 file3 #删除文件file1、file2、file3
rm -f file4 #强行删除文件file4
\end{verbatim}
\end{itemize}
\end{enumerate}
\item 目录
\begin{enumerate}
\item 创建目录
\item 复制目录
\item 移动目录
\item 删除目录
\end{enumerate}
\item 链接
\begin{enumerate}
\item 硬链接
\item 符号链接(软链接)
\end{enumerate}
\end{itemize}
\end{frame}
\begin{frame}
\frametitle{2. vi编辑器}
\label{sec-2-2}

\begin{itemize}
\item vi入门
\begin{enumerate}
\item 三种基本模式
\item 带参数启动与不带参数启动
\item 存盘退出与不存盘退出
\end{enumerate}
\item vi常用操作
\begin{enumerate}
\item 移动光标
\item 删除、复制、粘贴
\item 撤销与重做
\end{enumerate}
\item vi快速移动
\begin{enumerate}
\item 行首行尾
\item 上下翻页
\item 跳至第n行
\item 按语义单位移动(单词、句子、段落)
\item 次数前缀
\end{enumerate}
\item vi高效编辑
\begin{enumerate}
\item 行操作
\item 段操作
\item 块操作
\item 重复上次操作
\item 重复宏操作
\item 简写
\item 文件加密
\end{enumerate}
\item vi与多文件处理
\begin{enumerate}
\item 多文件
\item 多窗口
\item 多标签
\end{enumerate}
\item vi环境配置
\begin{enumerate}
\item 临时配置
\item 长期配置
\end{enumerate}
\end{itemize}
\end{frame}
\begin{frame}
\frametitle{3. 文本处理工具}
\label{sec-2-3}

\begin{itemize}
\item 查看文件内容
\begin{itemize}
\item cat、pr、more、less
\end{itemize}
\item 查看部分行
\begin{itemize}
\item head、tail、grep
\end{itemize}
\item 查看部分列
\begin{itemize}
\item cut
\end{itemize}
\item 消除相邻重复行
\begin{itemize}
\item uniq
\end{itemize}
\item 排序
\begin{itemize}
\item sort
\end{itemize}
\item 字符转换
\begin{itemize}
\item tr
\end{itemize}
\end{itemize}
\end{frame}
\begin{frame}
\frametitle{4. shell特性}
\label{sec-2-4}

\begin{itemize}
\item 命令组
\item 文件名通配符
\item 字符串扩展
\item I/O重定向
\item 管道
\end{itemize}
\end{frame}
\section{第3周}
\label{sec-3}
\begin{frame}
\frametitle{1. 用户管理}
\label{sec-3-1}

\begin{itemize}
\item 用户帐号
\item 用户数据文件
\end{itemize}
\end{frame}
\begin{frame}
\frametitle{2. 权限管理}
\label{sec-3-2}
\end{frame}
\begin{frame}
\frametitle{3. 进程管理}
\label{sec-3-3}
\end{frame}
\begin{frame}
\frametitle{4. shell环境}
\label{sec-3-4}
\end{frame}
\section{第4周}
\label{sec-4}
\begin{frame}
\frametitle{1. 正则表达式1}
\label{sec-4-1}
\end{frame}
\begin{frame}
\frametitle{2. 正则表达式2}
\label{sec-4-2}
\end{frame}
\begin{frame}
\frametitle{3. sed1}
\label{sec-4-3}
\end{frame}
\begin{frame}
\frametitle{4. sed2}
\label{sec-4-4}
\end{frame}
\section{第5周}
\label{sec-5}
\begin{frame}[fragile]
\frametitle{1. awk1}
\label{sec-5-1}
\begin{itemize}

\item 引言
\label{sec-5-1-1}%
\begin{itemize}

\item 计算机用户总是花大量时间用于处理简单的、机械性的数据处理工作,例如转换数据格式、查找包含特定属性的项目、汇总数据、打印报表等等。这些事情都应该机械化,但是,每当有这样的任务,就要用C或者Pascal语言写一个特殊任务的程序真是一件令人讨厌的事。AWK是一种编程语言,它使得利用很短的,常常是一两行的程序来完成这些任务成为可能。一段AWK程序就是一系列的模式和操作,它说明在输入数据中寻找什么以及找到之后干什么。AWK在文件中搜寻与任一模式匹配的行,当一个匹配的行被发现之后,对应的操作即被执行。一个模式可以通过联合正则表达式和比较运算对字符串,数字,字段,变量和数组元素做行选择。操作可以对选定的行进行任意的处理。操作语言看起来像C但是没有声明,而且字符串和数字是内建的数据类型。AWK扫描输入文件并且把每一输入行自动地分成字段。因为这么多事情是自动的--输入,字段分割,存储管理,和初始化--AWK程序通常都比它们在更常规的语言中短的多。因此AWK的一项通常的应用就是上述建议的数据操作。一两行的程序,由键盘键入,运行一次,然后丢掉。但事实上,AWK是一种普通的编程工具,它可以取代很多特殊的工具和程序。
\label{sec-5-1-1-1}%
\end{itemize} % ends low level

\item 入门指导
\label{sec-5-1-2}%
\begin{itemize}

\item 开始
\label{sec-5-1-2-1}%
\begin{itemize}

\item 引例:假设有一个文件emp.data,它包含了员工的姓名、以美元计的时薪以及工作小时数,每个员工记录占一行:\\
\label{sec-5-1-2-1-1}%
\begin{verbatim}
cat >emp.data <<EOF
Beth   4.00    0
Dan    3.75    0
Kathy  4.00    10
Mark   5.00    20
Mary   5.50    22
Susie  4.25    18
EOF
\end{verbatim}

现在需要打印每个工作时间大于0小时员工的姓名和应得薪酬(时薪x工作小时数)

\begin{verbatim}
awk '$3 > 0 { print $1, $2*$3 }' emp.data
\end{verbatim}

\item 如果要打印出从不工作的员工名单\\
\label{sec-5-1-2-1-2}%
\begin{verbatim}
awk '$3 == 0 { print $1 }' emp.data
\end{verbatim}

\item awk程序的结构
\label{sec-5-1-2-1-3}%
\begin{itemize}

\item 引例中写在引号内的部分是用awk语言写的程序。
\label{sec-5-1-2-1-3-1}%

\item 每个awk程序都是一系列模式-动作语句:
\label{sec-5-1-2-1-3-2}%
\end{itemize} % ends low level
\end{itemize} % ends low level

\item 简单输出
\label{sec-5-1-2-2}%

\item 格式化输出
\label{sec-5-1-2-3}%

\item 选择
\label{sec-5-1-2-4}%

\item 用awk计算
\label{sec-5-1-2-5}%

\item 流程控制语句
\label{sec-5-1-2-6}%

\item 数组
\label{sec-5-1-2-7}%

\item 单行awk
\label{sec-5-1-2-8}%
\end{itemize} % ends low level

\item awk语言
\label{sec-5-1-3}%
\begin{itemize}

\item 模式
\label{sec-5-1-3-1}%

\item 动作
\label{sec-5-1-3-2}%

\item 用户定义函数
\label{sec-5-1-3-3}%

\item 输出
\label{sec-5-1-3-4}%

\item 输入
\label{sec-5-1-3-5}%

\item 与其他程序交互
\label{sec-5-1-3-6}%
\end{itemize} % ends low level

\item 数据处理
\label{sec-5-1-4}%
\begin{itemize}

\item 数据转换与压缩
\label{sec-5-1-4-1}%

\item 数据可视化
\label{sec-5-1-4-2}%

\item bundle和unbundle
\label{sec-5-1-4-3}%

\item 多行记录
\label{sec-5-1-4-4}%
\end{itemize} % ends low level

\item 报表和数据库
\label{sec-5-1-5}%
\begin{itemize}

\item 生成报表
\label{sec-5-1-5-1}%

\item 包装查询和报表
\label{sec-5-1-5-2}%

\item 一个关系型数据库系统
\label{sec-5-1-5-3}%
\end{itemize} % ends low level

\item 处理文本
\label{sec-5-1-6}%
\begin{itemize}

\item 随机文本生成
\label{sec-5-1-6-1}%

\item 交互式文本操作
\label{sec-5-1-6-2}%

\item 文本处理
\label{sec-5-1-6-3}%
\end{itemize} % ends low level

\item 微型语言
\label{sec-5-1-7}%
\begin{itemize}

\item 一个汇编器和解释器
\label{sec-5-1-7-1}%

\item 一个画图语言
\label{sec-5-1-7-2}%

\item 一个排序生成器
\label{sec-5-1-7-3}%

\item 一个逆波兰计算器
\label{sec-5-1-7-4}%

\item 一个前缀计算机
\label{sec-5-1-7-5}%

\item 递归下降分析器
\label{sec-5-1-7-6}%
\end{itemize} % ends low level

\item 算法实验
\label{sec-5-1-8}%
\begin{itemize}

\item 排序
\label{sec-5-1-8-1}%

\item profiling
\label{sec-5-1-8-2}%

\item 拓扑排序
\label{sec-5-1-8-3}%

\item Make:一个文件更新程序
\label{sec-5-1-8-4}%
\end{itemize} % ends low level

\item 后记
\label{sec-5-1-9}%
\begin{itemize}

\item AWK作为一种语言
\label{sec-5-1-9-1}%

\item 性能
\label{sec-5-1-9-2}%

\item 结论
\label{sec-5-1-9-3}%
\end{itemize} % ends low level
\end{itemize} % ends low level
\end{frame}
\begin{frame}
\frametitle{2. awk2}
\label{sec-5-2}
\end{frame}
\begin{frame}[fragile]
\frametitle{3. shell编程1}
\label{sec-5-3}
\begin{itemize}

\item 3.1 创建新命令
\label{sec-5-3-1}%

\item 3.2 命令参数
\label{sec-5-3-2}%

\item 3.3 shell变量
\label{sec-5-3-3}%

\item 3.4 test
\label{sec-5-3-4}%

\item 3.5 流程控制语句
\label{sec-5-3-5}%

\item 3.6 数据重定向
\label{sec-5-3-6}%

\item 3.7 管道输入读
\label{sec-5-3-7}%

\item 3.8 命令行选项
\label{sec-5-3-8}%

\item 3.9 计算
\label{sec-5-3-9}%

\item 3.10 函数
\label{sec-5-3-10}%

\item 3.11 中断处理
\label{sec-5-3-11}%

\item 3.12 调试
\label{sec-5-3-12}%
\begin{itemize}
\item set命令
\begin{itemize}
\item 在脚本中添加命令set -x,在该命令之后执行的每条命令及其参数和参数的值都会显示出来。
\item set -v,只显示正在运行的脚本的代码。
\item set +x/+v,关闭x/v选项。
\end{itemize}
\item echo/print命令
\item 根据调试层次控制输出
\begin{itemize}
\item 在脚本开始处设置一个调试变量,脚本运行时测试该变量,然后根据该变量的值决定是否启用调试指令。

\begin{verbatim}
debug=1
test debug -gt 0 && echo "debug is on"
\end{verbatim}
\end{itemize}
\item 用函数简化错误检查

\begin{verbatim}
alert(){
  # usage: alert <$?> <object>
  if [ "$1" -ne 0 ]; then
    echo "WARNING: $2 did not complete successfully." >&2
    exit $1
  else
    echo "INFO: $2 complete successfully" >&2
  fi
}
\end{verbatim}

\begin{verbatim}
alert(){
  local RET_CODE=$?
  if [ -z "$DEBUG" ] || [ "$DEBUG" -eq 0 ]; then
    return
  fi
  if [ "$RET_CODE" -ne 0 ]; then
    echo "Warn: $* failed with a return code of $RET_CODE." >&2
    [ "$DEBUG" -gt 9 ] && exit "$RET_CODE"
    [ "$STEP_THROUGH" = 1 ] && {
      echo "Press [Enter] to continue" >&2; read x
    }
  fi
  [ "$STEP_THROUGH" = 2 ] && {
    echo "Press [Enter] to continue" >&2; read x
  }
}
\end{verbatim}
\end{itemize}

\item 3.13 函数库
\label{sec-5-3-13}%
\begin{itemize}
\item 库:包含在单个文件中的一个函数集。
\item 例:

\begin{verbatim}
evenodd(){
  # 根据最后一位数字确定奇偶性
  LAST_DIGIT=`echo $1 | sed 's/\(.*\)\(.\)$/\2/'`
  case $LAST_DIGIT in
  0|2|4|6|8 )
    return 1
  ;;
  *)
    return 0
  ;;
  esac
}
\end{verbatim}

\begin{verbatim}
isalive(){
  #判断一台远程机器是否能ping通
  NODE=$1
  $PING -c 3 $NODE >/dev/null 2>&1
  if [ $? -eq 0 ]; then
    return 1
  else
    return 0
  fi
}
\end{verbatim}
\item 使用库

\begin{verbatim}
source std_lib
\end{verbatim}

\begin{verbatim}
. std_lib
\end{verbatim}
\end{itemize}
\end{itemize} % ends low level
\end{frame}
\begin{frame}
\frametitle{4. shell编程2}
\label{sec-5-4}
\end{frame}
\section{第6周}
\label{sec-6}
\begin{frame}
\frametitle{1. shell编程3}
\label{sec-6-1}
\end{frame}
\begin{frame}
\frametitle{2. shell编程4}
\label{sec-6-2}
\end{frame}
\begin{frame}
\frametitle{3. 磁盘与文件系统管理1}
\label{sec-6-3}
\end{frame}
\begin{frame}
\frametitle{4. 磁盘与文件系统管理2}
\label{sec-6-4}
\end{frame}
\section{第7周}
\label{sec-7}
\begin{frame}
\frametitle{1. 磁盘阵列管理1}
\label{sec-7-1}
\end{frame}
\begin{frame}
\frametitle{2. 磁盘阵列管理2}
\label{sec-7-2}
\end{frame}
\begin{frame}
\frametitle{3. 磁盘配额管理}
\label{sec-7-3}
\end{frame}
\begin{frame}
\frametitle{4. 逻辑卷管理}
\label{sec-7-4}
\end{frame}
\section{第8周}
\label{sec-8}
\begin{frame}
\frametitle{1. Linux软件包管理1}
\label{sec-8-1}
\end{frame}
\begin{frame}
\frametitle{2. Linux软件包管理2}
\label{sec-8-2}
\end{frame}
\begin{frame}
\frametitle{3. Linux内核管理与硬件管理}
\label{sec-8-3}
\end{frame}
\begin{frame}
\frametitle{4. Linux启动管理1}
\label{sec-8-4}
\end{frame}
\section{第9周}
\label{sec-9}
\begin{frame}
\frametitle{1. Linux启动管理2}
\label{sec-9-1}
\end{frame}
\begin{frame}
\frametitle{2. Linux启动管理3}
\label{sec-9-2}
\end{frame}
\begin{frame}
\frametitle{3. Linux服务管理}
\label{sec-9-3}
\end{frame}
\begin{frame}
\frametitle{4. Linux故障诊断与恢复}
\label{sec-9-4}
\end{frame}
\section{第10周}
\label{sec-10}
\begin{frame}
\frametitle{1. Linux性能检测}
\label{sec-10-1}
\end{frame}
\begin{frame}
\frametitle{2. Linux日志管理}
\label{sec-10-2}
\end{frame}
\begin{frame}
\frametitle{3. Linux网络基本配置}
\label{sec-10-3}
\end{frame}
\begin{frame}
\frametitle{4. PAM认证、TCP wrapper、Xinetd访问控制与主机防火墙}
\label{sec-10-4}
\end{frame}
\section{第11周}
\label{sec-11}
\begin{frame}
\frametitle{1. DNS服务配置1}
\label{sec-11-1}
\end{frame}
\begin{frame}
\frametitle{2. DNS服务配置2}
\label{sec-11-2}
\end{frame}
\begin{frame}
\frametitle{3. DHCP服务配置1}
\label{sec-11-3}
\end{frame}
\begin{frame}
\frametitle{4. DHCP服务配置2}
\label{sec-11-4}
\end{frame}
\section{第12周}
\label{sec-12}
\begin{frame}
\frametitle{1. NFS服务配置1}
\label{sec-12-1}
\end{frame}
\begin{frame}
\frametitle{2. NFS服务配置2}
\label{sec-12-2}
\end{frame}
\begin{frame}
\frametitle{3. samba服务配置1}
\label{sec-12-3}
\end{frame}
\begin{frame}
\frametitle{4. samba服务配置2}
\label{sec-12-4}
\end{frame}
\section{第13周}
\label{sec-13}
\begin{frame}
\frametitle{1. ftp服务配置1}
\label{sec-13-1}
\end{frame}
\begin{frame}
\frametitle{2. ftp服务配置2}
\label{sec-13-2}
\end{frame}
\begin{frame}
\frametitle{3. apache服务配置1}
\label{sec-13-3}
\end{frame}
\begin{frame}
\frametitle{4. apache服务配置2}
\label{sec-13-4}
\end{frame}
\section{第14周}
\label{sec-14}
\begin{frame}
\frametitle{1. 远程访问配置1}
\label{sec-14-1}
\end{frame}
\begin{frame}
\frametitle{2. 远程访问配置2}
\label{sec-14-2}
\end{frame}
\begin{frame}
\frametitle{3. 防火墙1}
\label{sec-14-3}
\end{frame}
\begin{frame}
\frametitle{4. 防火墙2}
\label{sec-14-4}
\end{frame}
\section{第15周}
\label{sec-15}
\begin{frame}
\frametitle{1. squid代理服务配置1}
\label{sec-15-1}
\end{frame}
\begin{frame}
\frametitle{2. squid代理服务配置2}
\label{sec-15-2}
\end{frame}
\begin{frame}
\frametitle{3. Linux安全1}
\label{sec-15-3}
\end{frame}
\begin{frame}
\frametitle{4. Linux安全2}
\label{sec-15-4}
\end{frame}
\section{第16周}
\label{sec-16}
\begin{frame}
\frametitle{1. 复习1}
\label{sec-16-1}
\end{frame}
\begin{frame}
\frametitle{2. 复习2}
\label{sec-16-2}
\end{frame}
\begin{frame}
\frametitle{3. 期末上机考试1}
\label{sec-16-3}
\end{frame}
\begin{frame}
\frametitle{4. 期末上机考试2}
\label{sec-16-4}
\end{frame}

\end{document}
