% Created 2016-03-13 日 13:11
\documentclass[xcolor=svgnames,presentation]{beamer}
\usepackage[utf8]{inputenc}
\usepackage[T1]{fontenc}
\usepackage{fixltx2e}
\usepackage{graphicx}
\usepackage{longtable}
\usepackage{float}
\usepackage{wrapfig}
\usepackage{soul}
\usepackage{textcomp}
\usepackage{marvosym}
\usepackage{wasysym}
\usepackage{latexsym}
\usepackage{amssymb}
\usepackage{hyperref}
\tolerance=1000
\usepackage{minted}
\usecolortheme[named=FireBrick]{structure}\setbeamercovered{transparent}\setbeamertemplate{caption}[numbered]\setbeamertemplate{blocks}[rounded][shadow=true] \usetheme{Darmstadt}\date{\today} \usepackage{tikz}\usepackage{xeCJK}\usepackage{amsmath}\setmainfont{Times New Roman}\setCJKmainfont[BoldFont={Adobe Heiti Std},ItalicFont={Adobe Fangsong Std}]{Adobe Heiti Std}\setCJKsansfont{Adobe Heiti Std}\setCJKmonofont{Adobe Fangsong Std}\usepackage{verbatim}\graphicspath{{figures/}} \definecolor{lstbgcolor}{rgb}{0.9,0.9,0.9} \usepackage{listings}\usepackage{minted} \usepackage{fancyvrb}\usepackage{xcolor}\lstset{escapeinside=`',frameround=ftft,language=C,breaklines=true,keywordstyle=\color{blue!70},commentstyle=\color{red!50!green!50!blue!50},frame=shadowbox,backgroundcolor=\color{yellow!20},rulesepcolor=\color{red!20!green!20!blue!20}}
\usemintedstyle{default}
\providecommand{\alert}[1]{\textbf{#1}}

\title{第3讲 Linux基础(2)}
\author{王晓庆}
\date{\today}
\hypersetup{
  pdfkeywords={},
  pdfsubject={},
  pdfcreator={Emacs Org-mode version 7.9.3f}}

\institute{wangxiaoqing@outlook.com}
\begin{document}

\maketitle

\begin{frame}
\frametitle{Outline}
\setcounter{tocdepth}{1}
\tableofcontents
\end{frame}

\section{用户和组管理}
\label{sec-1}
\subsection{帐号分类}
\label{sec-1-1}
\begin{frame}
\frametitle{帐号}
\label{sec-1-1-1}
\begin{itemize}

\item 帐号(account)是用于记录用户或用户组的数据
\label{sec-1-1-1-1}%

\item Linux中每一个合法用户都必须拥有帐号才能登录使用系统
\label{sec-1-1-1-2}%

\item Linux依靠帐号来验证用户身份并设置用户权限
\label{sec-1-1-1-3}%
\end{itemize} % ends low level
\end{frame}
\begin{frame}
\frametitle{用户帐号和组帐号}
\label{sec-1-1-2}
\begin{itemize}

\item 用户帐号:每个用户帐号存储某个用户的数据
\label{sec-1-1-2-1}%

\item 组帐号:每个组帐号存储某个用户组的数据
\label{sec-1-1-2-2}%

\item CentOS中最多可以建立2**30个帐号和2**30个组帐号
\label{sec-1-1-2-3}%
\end{itemize} % ends low level
\end{frame}
\begin{frame}
\frametitle{根据帐号位置分类}
\label{sec-1-1-3}
\begin{itemize}

\item 本机帐号(local account)
\label{sec-1-1-3-1}%
\begin{itemize}

\item 帐号数据存储于本机硬盘内
\label{sec-1-1-3-1-1}%

\item 帐号有效范围仅限于本机
\label{sec-1-1-3-1-2}%

\item 优点:简单易用,无需额外设置即可直接创建
\label{sec-1-1-3-1-3}%

\item 缺点:不具备可扩展性,设想一个数百台设备的环境,要让某个帐号在每台设备上都能登录,则必须在每台设备上建立该帐号
\label{sec-1-1-3-1-4}%
\end{itemize} % ends low level

\item 域帐号(domain account)
\label{sec-1-1-3-2}%
\begin{itemize}

\item 将大量计算机组织成一个域,并在其中一台设备上建立帐号数据,并通过某些通信协议实现帐号数据的远程访问
\label{sec-1-1-3-2-1}%

\item 优点:具备可扩展性
\label{sec-1-1-3-2-2}%

\item 缺点:配置域帐号必须要先建立域环境
\label{sec-1-1-3-2-3}%
\end{itemize} % ends low level
\end{itemize} % ends low level
\end{frame}
\begin{frame}
\frametitle{根据帐号功能分类}
\label{sec-1-1-4}
\begin{itemize}

\item 用户帐号
\label{sec-1-1-4-1}%
\begin{itemize}

\item 超级用户:用户名通常为root,其UID一定为0
\label{sec-1-1-4-1-1}%
\begin{itemize}

\item 权限不受任何限制
\label{sec-1-1-4-1-1-1}%

\item 不要轻易以root用户身份使用Linux系统。
\label{sec-1-1-4-1-1-2}%
\end{itemize} % ends low level

\item 普通用户:除超级用户外的其他所有帐号
\label{sec-1-1-4-1-2}%
\begin{itemize}

\item 系统帐号
\label{sec-1-1-4-1-2-1}%
\begin{itemize}

\item 仅提供给Linux系统本身使用,某些软件运行时需要一个普通用户身份
\label{sec-1-1-4-1-2-1-1}%

\item 在CentOS 5中,系统用户的UID为1\~{}499之间
\label{sec-1-1-4-1-2-1-2}%
\end{itemize} % ends low level

\item 真实用户
\label{sec-1-1-4-1-2-2}%
\begin{itemize}

\item 真实用户就是可以用于登录系统的普通帐号
\label{sec-1-1-4-1-2-2-1}%

\item 真实用户的UID在500~4,294,967,295之间
\label{sec-1-1-4-1-2-2-2}%
\end{itemize} % ends low level
\end{itemize} % ends low level
\end{itemize} % ends low level
\end{itemize} % ends low level
\end{frame}
\begin{frame}
\frametitle{根据帐号功能分类}
\label{sec-1-1-5}
\begin{itemize}

\item 组帐号
\label{sec-1-1-5-1}%
\begin{itemize}

\item 超级用户组
\label{sec-1-1-5-1-1}%
\begin{itemize}

\item root组,GID为0,但组成员不具备系统管理权力,受系统权限限制
\label{sec-1-1-5-1-1-1}%
\end{itemize} % ends low level

\item 系统组
\label{sec-1-1-5-1-2}%
\begin{itemize}

\item 类似于系统帐号,仅提供给系统本身或某个软件使用
\label{sec-1-1-5-1-2-1}%

\item 在CentOS 5中,系统组的GID范围为1~499
\label{sec-1-1-5-1-2-2}%
\end{itemize} % ends low level

\item 用户自定义组
\label{sec-1-1-5-1-3}%
\begin{itemize}

\item 除root组和系统组之外的组,都可以由管理员自行定义
\label{sec-1-1-5-1-3-1}%

\item 用户自定义组的GID在500~4,294,967,295之间
\label{sec-1-1-5-1-3-2}%

\item 用户私有组(User Private Group, UPG)
\label{sec-1-1-5-1-3-3}%
\begin{enumerate}
\item 是指与用户账户名称相同的组,是用户默认的基本组
\item 当建立新用户时,CentOS会自动建立该用户的私有组
\item 用户还可以加入多个其他的组,也叫用户的附加组
\end{enumerate}
\end{itemize} % ends low level
\end{itemize} % ends low level
\end{itemize} % ends low level
\end{frame}
\subsection{帐号内容}
\label{sec-1-2}
\begin{frame}[fragile]
\frametitle{用户帐号}
\label{sec-1-2-1}
\begin{itemize}

\item 用户帐号数据和密码分别存储于/etc/passwd和/etc/shadow中\\
\label{sec-1-2-1-1}%
\begin{minted}[]{bash}
less /etc/passwd
less /etc/shadow
man 5 passwd
man 5 shadow
\end{minted}
\end{itemize} % ends low level
\begin{exampleblock}{说明}
\label{sec-1-2-1-2}

\begin{enumerate}
\item 如果密码字段为空,则该用户无需密码即可登录系统
\item 如果密码字段为x,则该用户名密码被存储至他处
\item 如果默认shell为/sbin/nologin,则不允许该用户登录
\end{enumerate}
\end{exampleblock}
\end{frame}
\begin{frame}[fragile]
\frametitle{组帐号}
\label{sec-1-2-2}
\begin{itemize}

\item 组帐号数据和密码分别存储于/etc/group和/etc/gshadow中\\
\label{sec-1-2-2-1}%
\begin{minted}[]{bash}
less /etc/group
less /etc/gshadow
man 5 group
man 5 gshadow
\end{minted}
\end{itemize} % ends low level
\begin{exampleblock}{说明}
\label{sec-1-2-2-2}

\begin{enumerate}
\item CentOS不支持嵌套组,组成员只能是用户,不能是其他组
\item 组成员之间以逗号(,)分隔
\end{enumerate}
\end{exampleblock}
\end{frame}
\subsection{用户帐号管理}
\label{sec-1-3}
\begin{frame}[fragile]
\frametitle{添加用户帐号}
\label{sec-1-3-1}
\begin{block}{useradd}
\label{sec-1-3-1-1}


\begin{minted}[]{bash}
useradd jack
tail -1 /etc/passwd
tail -1 /etc/group
tail -1 /etc/shadow
ls /home
ls /var/spool/mail
man useradd
\end{minted}
\end{block}
\begin{exampleblock}{说明}
\label{sec-1-3-1-2}

用useradd添加新用户时,如果未指定选项,则useradd会根据/etc/login.defs与/etc/default/useradd中的配置添加新用户。
\end{exampleblock}
\end{frame}
\begin{frame}[fragile]
\frametitle{设置/修改用户密码}
\label{sec-1-3-2}
\begin{exampleblock}{示例}
\label{sec-1-3-2-1}


\begin{minted}[]{bash}
passwd jack    #为指定用户设置密码(仅root)
passwd         #修改当前用户密码
passwd -l jack #锁定用户(仅root)
passwd -u jack #解锁用户(仅root)
passwd -d jack #清除用户密码(本地免密码登录)
\end{minted}
\end{exampleblock}
\begin{block}{说明}
\label{sec-1-3-2-2}

让用户免密码本地登录的两种方法:
\begin{enumerate}
\item 将/etc/passwd文件中的密码字段清空
\item 将/etc/shadow文件中的密码字段清空
\end{enumerate}
远程用户必须提供密码才允许登录!
\end{block}
\end{frame}
\begin{frame}[fragile]
\frametitle{修改用户帐号}
\label{sec-1-3-3}
\begin{block}{usermod}
\label{sec-1-3-3-1}


\begin{minted}[]{bash}
usermod -u 1001 bob
usermod -g 1001 bob
usermod -G root,users bob #设置bob的附加组
usermod -G '' bob         #设置bob不属于任何附加组
usermod -s /bin/csh bob
usermod -d /home/bobie
usermod -l bobie bob
usermod -L alice          #锁定帐号
usermod -U alice          #解锁帐号
\end{minted}
\end{block}
\begin{exampleblock}{说明}
\label{sec-1-3-3-2}

usermod与useradd的选项类似,还增加了-U、-L、-l等选项。
\end{exampleblock}
\end{frame}
\begin{frame}[fragile]
\frametitle{查看和删除用户帐号}
\label{sec-1-3-4}
\begin{exampleblock}{id}
\label{sec-1-3-4-1}


\begin{minted}[]{bash}
id          #查看当前用户信息
id jack     #查看指定用户信息
\end{minted}
\end{exampleblock}
\begin{block}{userdel}
\label{sec-1-3-4-2}


\begin{minted}[]{bash}
userdel jack     #删除用户
userdel -r jack  #删除用户及其主目录和邮箱
\end{minted}
\end{block}
\end{frame}
\subsection{组帐号管理}
\label{sec-1-4}
\begin{frame}[fragile]
\frametitle{创建、修改组}
\label{sec-1-4-1}
\begin{exampleblock}{groupadd 创建组}
\label{sec-1-4-1-1}


\begin{minted}[]{bash}
groupadd proj1
groupadd -g 1001 proj2
\end{minted}
\end{exampleblock}
\begin{exampleblock}{groupmod 修改组}
\label{sec-1-4-1-2}


\begin{minted}[]{bash}
groupmod -g 1000 students
groupmod -n projectA proj1
\end{minted}
\end{exampleblock}
\end{frame}
\begin{frame}[fragile]
\frametitle{查询、删除组}
\label{sec-1-4-2}
\begin{exampleblock}{groups 查询组}
\label{sec-1-4-2-1}


\begin{minted}[]{bash}
groups
groups mary
\end{minted}
\end{exampleblock}
\begin{exampleblock}{groupdel 删除组}
\label{sec-1-4-2-2}


\begin{minted}[]{bash}
groupdel projectA
\end{minted}
\end{exampleblock}
\begin{block}{注意}
\label{sec-1-4-2-3}


\begin{verbatim}
注意:删除某用户的基本组之前,需要先删除该用户!
\end{verbatim}
\end{block}
\end{frame}
\begin{frame}[fragile]
\frametitle{组成员管理}
\label{sec-1-4-3}
\begin{itemize}

\item gpasswd
\label{sec-1-4-3-1}%
\begin{itemize}

\item gpasswd本用于设置组密码,但该功能极少使用,实际上更多用于组成员管理
\label{sec-1-4-3-1-1}%
\end{itemize} % ends low level
\end{itemize} % ends low level
\begin{exampleblock}{示例}
\label{sec-1-4-3-2}


\begin{minted}[]{bash}
gpaswd -a mary students   #将mary加入students组
gpasswd -d mary students  #从students组删除mary
gpasswd -A mary students  #将mary设置为studens组管理员
\end{minted}
\end{exampleblock}
\end{frame}
\begin{frame}[fragile]
\frametitle{切换组身份}
\label{sec-1-4-4}
\begin{exampleblock}{newgrp}
\label{sec-1-4-4-1}


\begin{minted}[]{bash}
newgrp students  #切换至students组身份
exit             #返回原先组身份
\end{minted}
\end{exampleblock}
\begin{block}{说明}
\label{sec-1-4-4-2}


\begin{verbatim}
如果用户不是root,下列情况下将会被提示输入组密码:
1. 如果用户没有密码而要切换的组有密码
2. 如果用户不属于要切换的组,而该组有密码
如果用户不属于要切换的组,且该组未设密码,则拒绝切换
\end{verbatim}
\end{block}
\note{sg


\begin{verbatim}
man sg   #sg与newgrp类似,但也有些不同
\end{verbatim}
}
\end{frame}
\subsection{密码管理}
\label{sec-1-5}
\begin{frame}
\frametitle{加密算法}
\label{sec-1-5-1}
\begin{itemize}

\item DES\\
\label{sec-1-5-1-1}%
传统UNIX使用的加密算法,只能支持8个字符以内的密码,如果超过8个字符,则DES会忽略第8个字符以后的密码,其安全性比较差。

\item MD5\\
\label{sec-1-5-1-2}%
CentOS默认使用的加密算法,支持255个字符的密码,比DES更安全。
\end{itemize} % ends low level
\end{frame}
\begin{frame}[fragile]
\frametitle{查看与设置帐号秘密有效期}
\label{sec-1-5-2}
\begin{exampleblock}{chage}
\label{sec-1-5-2-1}


\begin{minted}[]{bash}
chage -l mike  #查看用户帐号密码有效期信息
#注意:普通用户只能查看自己的帐号密码有效期信息
chage -m 3 mike  #设置设定密码后至少经过几天才能修改密码
chage -M 30 mike #设置设定密码后最长多少天后必须修改密码
chage -d 2016-01-01 mike #设置密码最后修改时间
chage -E 2020-12-31 mike #设置密码过期时间
chage -W 7 mike  #设置密码过期前几天提醒
chage -I 3 mike  #设置密码过期后多少天被锁定
\end{minted}
\end{exampleblock}
\end{frame}
\section{权限管理}
\label{sec-2}
\subsection{基本权限管理}
\label{sec-2-1}
\begin{frame}
\frametitle{文件权限和目录权限}
\label{sec-2-1-1}
\begin{itemize}

\item 三种基本权限\\
\label{sec-2-1-1-1}%
\begin{center}
\begin{tabular}{lll}
 r--       &  -w-       &  --x         \\
\hline
 100       &  010       &  001         \\
 4         &  2         &  1           \\
 文件可读  &  文件可写  &  文件可执行  \\
 目录可读  &  目录可写  &  目录可搜索  \\
\end{tabular}
\end{center}


\end{itemize} % ends low level
\begin{block}{说明}
\label{sec-2-1-1-2}

\begin{enumerate}
\item 目录可读:可以通过ls查看目录中的文件列表
\item 目录可写:可以在目录内创建、删除、移动文件
\item 目录可访问:可以通过cd进入该目录或者搜索其内的文件
\end{enumerate}
\end{block}
\end{frame}
\begin{frame}[fragile]
\frametitle{三种用户}
\label{sec-2-1-2}
\begin{exampleblock}{三种用户的权限}
\label{sec-2-1-2-1}


\begin{minted}[]{bash}
ls -l ~/students.db
\end{minted}
-rw-rw-r--. 1 wxq wxq 115 3月   8 22:35 /home/wxq/students.db
\end{exampleblock}
\begin{itemize}

\item 文件所有者(默认为创建该文件的用户)\\
\label{sec-2-1-2-2}%
ls -l输出结果第3列,第一组权限(rw-)

\item 文件所属组\\
\label{sec-2-1-2-3}%
ls -l输出结果第4列,第二组权限(rw-)

\item 其他用户\\
\label{sec-2-1-2-4}%
既不是文件所有者,也不属于文件所属组,第三组权限(r--)
\end{itemize} % ends low level
\end{frame}
\begin{frame}[fragile]
\frametitle{修改文件和目录权限}
\label{sec-2-1-3}
\begin{itemize}

\item chmod\\
\label{sec-2-1-3-1}%
仅文件所有者和root用户有权修改文件权限
\end{itemize} % ends low level
\begin{exampleblock}{示例}
\label{sec-2-1-3-2}


\begin{minted}[]{bash}
chmod u=rw-,g=r--,o=r-- file
chmod g+w,o-r file
chmod a+x file
chmod 644 file
chmod 600 file
chmod 700 dir
chmod -R 755 dir
\end{minted}
\end{exampleblock}
\end{frame}
\begin{frame}[fragile]
\frametitle{查看和设置默认权限掩码}
\label{sec-2-1-4}
\begin{exampleblock}{umask}
\label{sec-2-1-4-1}


\begin{minted}[]{bash}
umask
umask -S
umask 0026
\end{minted}
\end{exampleblock}
\begin{block}{说明}
\label{sec-2-1-4-2}

\begin{enumerate}
\item 新建文件的默认权限等于0666-umask
\item 新建目录的默认权限等于0777-umask
\item 0666-0027=0640
\end{enumerate}
\end{block}
\end{frame}
\begin{frame}[fragile]
\frametitle{权限设置实例}
\label{sec-2-1-5}
\begin{block}{问题}
\label{sec-2-1-5-1}

xiaobai想在自己的主目录内建立一个公共目录pub,希望所有人都可以只读访问该目录内的文件,但不希望其他人能访问其主目录内的其他文件和目录,请问要怎么配置相关文件/目录的权限才能达到目的呢?
\end{block}
\note{答案


\begin{verbatim}
1. chmod o+x ~xiaobai
2. mkdir pub
3. echo hello >pub/hello.txt
\end{verbatim}
}
\note{另一个问题

mike想在自己的主目录内建立一个作业上传目录handin,希望所有学生都可以上传作业到该目录内,但不希望学生能访问其主目录内的其他文件和目录,也不希望学生访问/删除handin目录内其他同学的作业,请问要怎么才能做到呢?
}
\note{答案


\begin{verbatim}
1. chmod o+x ~mike
2. mkdir handin
3. chmod 1707 handin #设置其他用户可写,并设置sticky位
4. umask 0077        #设置默认权限? 这样不行,估计只能通过ftp的local_mask实现
\end{verbatim}
}
\end{frame}
\subsection{文件所有权管理}
\label{sec-2-2}
\begin{frame}[fragile]
\frametitle{chown和chgrp}
\label{sec-2-2-1}
\begin{itemize}

\item 仅root用户有权更改文件所有者和所属组
\label{sec-2-2-1-1}%
\end{itemize} % ends low level
\begin{exampleblock}{示例}
\label{sec-2-2-1-2}


\begin{minted}[]{bash}
chown root file         #令file为root所有
chown root:users file   #令file为root所有,属users组
chown root: file        #令file为root所有,属root组
chown :users file       #令file属users组
chown -R root:users dir #递归处理
chgrp users file        #令file属users组
chgrp -R users dir      #递归处理
\end{minted}
\end{exampleblock}
\end{frame}
\subsection{特殊权限(s)管理}
\label{sec-2-3}
\begin{frame}
\frametitle{三种基本权限够用吗?}
\label{sec-2-3-1}
\begin{itemize}

\item mike用户对文件F具有只读权限,则mike用编辑器vim打开文件F时,vim对文件F也仅具备只读权限。
\label{sec-2-3-1-1}%
\end{itemize} % ends low level
\begin{exampleblock}{结论}
\label{sec-2-3-1-2}

一个程序对文件具备何种权限由启动该程序的用户决定
\end{exampleblock}
\begin{block}{问题}
\label{sec-2-3-1-3}

mike用户对/etc/shadow不具备任何权限,那mike如何通过执行passwd程序修改存放在/etc/shadow内的登录密码呢?
\end{block}
\end{frame}
\begin{frame}[fragile]
\frametitle{setUID权限}
\label{sec-2-3-2}
\begin{itemize}

\item setUID权限:可执行程序被设置了setUID权限后,无论被谁执行,该程序都将具备其所有者的权限,而不是执行者的权限。
\label{sec-2-3-2-1}%
\end{itemize} % ends low level
\begin{exampleblock}{示例}
\label{sec-2-3-2-2}


\begin{minted}[]{bash}
ls -l /usr/bin/passwd
chmod u+s file   #设置setUID权限
chmod u-s file   #删除setUID权限
chmod 4751 file  #设置setUID权限
chmod 0751 file  #删除setUID权限
\end{minted}
\end{exampleblock}
\end{frame}
\begin{frame}[fragile]
\frametitle{setGID权限}
\label{sec-2-3-3}
\begin{itemize}

\item setGID权限:可执行程序被设置了setGID权限后,无论被谁执行,该程序都将具备其所属组的权限,但没有其所有者的权限。
\label{sec-2-3-3-1}%
\end{itemize} % ends low level
\begin{exampleblock}{示例}
\label{sec-2-3-3-2}


\begin{minted}[]{bash}
ls -l /bin/locate
chmod g+s file
chmod g-s file
chmod 2751 file
\end{minted}
\end{exampleblock}
\end{frame}
\begin{frame}[fragile]
\frametitle{setGID权限(2)}
\label{sec-2-3-4}
\begin{itemize}

\item setGID权限常针对目录设置:目录设置了setGID权限后,无论谁在该目录内创建的新文件,默认都会属于该目录所属的组,而不是创建者所属的组。
\label{sec-2-3-4-1}%
\end{itemize} % ends low level
\begin{exampleblock}{示例}
\label{sec-2-3-4-2}


\begin{minted}[]{bash}
mkdir proj_dir
groupadd proj
chgrp proj proj_dir
chmod g+s proj_dir
\end{minted}
\end{exampleblock}
\end{frame}
\begin{frame}
\frametitle{Sticky bit权限}
\label{sec-2-3-5}
\begin{itemize}

\item 
%
\end{itemize} % ends low level
\end{frame}
\subsection{}
\section{进程管理}
\label{sec-3}
\subsection{进程管理}
\label{sec-3-1}
\begin{frame}[fragile]
\frametitle{查看进程}
\label{sec-3-1-1}
\begin{exampleblock}{ps}
\label{sec-3-1-1-1}


\begin{minted}[]{bash}
ps             #查看当前shell执行的进程
ps aux         #查看所有进程(BSD风格)
ps -ef         #查看所有进程(System V风格)
ps -ft /dev/pts/0 #查看与指定终端关联的进程
ps -fp 1234    #查看进程号为1234的进程的详细信息
ps -fu xiaobai #查看用户xiaobai运行的进程信息
ps -fC evince  #查看名称为enince的所有进程信息
\end{minted}
\end{exampleblock}
\begin{block}{pstree}
\label{sec-3-1-1-2}


\begin{minted}[]{bash}
pstree         #打印系统进程树
\end{minted}
\end{block}
\end{frame}
\begin{frame}[fragile]
\frametitle{查看进程}
\label{sec-3-1-2}
\begin{block}{pgrep}
\label{sec-3-1-2-1}


\begin{minted}[]{bash}
pgrep evince  #查找名为evince的进程的进程号
\end{minted}
\end{block}
\begin{exampleblock}{top}
\label{sec-3-1-2-2}


\begin{minted}[]{bash}
top     #实时系统状态监控
h/?     #帮助
d 5     #设置刷新周期为5秒
k 1234  #杀死进程号为1234的进程
P       #按cpu使用率排序
M       #按memory使用率排序
T       #按cpu使用时间排序
u tom   #仅显示用户tom的进程
q       #退出top
man top
\end{minted}
\end{exampleblock}
\end{frame}
\begin{frame}[fragile]
\frametitle{进程控制}
\label{sec-3-1-3}
\begin{itemize}

\item 前后台进程
\label{sec-3-1-3-1}%
\begin{itemize}

\item 一个终端一次只能运行一个前台进程,但可以并发运行多个后台进程。
\label{sec-3-1-3-1-1}%
\end{itemize} % ends low level
\begin{exampleblock}{前后台作业控制}
\label{sec-3-1-3-1-2}


\begin{minted}[]{bash}
Ctrl-z              #挂起(暂停)前台进程的执行
jobs [-l]           #查看作业
fg [n]              #将作业恢复至前台执行
bg [n]              #将作业恢复至后台执行
kill %n/pid         #杀死指定的作业/进程
killall xeyes       #杀死所有xeyes进程
pkill eye           #查找包含eye的进程名并杀死
xeyes -center red & #启动后台进程
nohup ./longjob &   #保持后台进程在用户注销后继续运行
nohup ./longjob & >longjob.out
\end{minted}
\end{exampleblock}
\end{itemize} % ends low level
\end{frame}
\begin{frame}
\frametitle{进程和信号}
\label{sec-3-1-4}
\begin{itemize}

\item 信号\\
\label{sec-3-1-4-1}%
内核用信号通知进程发生的异常事件和实现进程间通信,进程也可以给其他进程。每种信号可以用名称或整数标识。
\begin{enumerate}
\item 硬件异常:进程让硬件执行错误操作,如进程进行除0运算,内核会给它发送SIGFPE(8)
\item 软件状态:把异常的软件状态通知进程,如进程终止时,内核会向父进程发送SIGCHLD(17),当调整X图形应用程序的窗口大小时,该程序会收到SIGWINCH(28)
\item 终端中断:用户敲入的终端控制组合键会把信号发送给shell前台进程,如Ctrl-c发送SIGINT(2),Ctrl-z发送SIGSTP(20)
\item 其他进程:进程可以通过kill命令给同一用户拥有的其他任何进程发送任何信号
\end{enumerate}
\end{itemize} % ends low level
\end{frame}
\begin{frame}[fragile]
\frametitle{向进程发信号}
\label{sec-3-1-5}
\begin{itemize}

\item 信号:
\label{sec-3-1-5-1}%
\end{itemize} % ends low level
\begin{exampleblock}{kill 发送信号给进程}
\label{sec-3-1-5-2}


\begin{minted}[]{bash}
kill -l              #打印信号值列表
man 7 signal         #查看信号相关帮助
Ctrl-c               #发送信号2给前台进程
Ctrl-z               #发送信号20给前台进程
kill -19 12121       #暂停
kill -18 12121       #恢复
kill -15 12345       #发送信号15给进程12345
kill 12345           #同上
kill -9 11111        #强行杀死进程11111
kill -s SIGTERM 1234 #发送信号SIGTERM给进程1234
\end{minted}
\end{exampleblock}
\end{frame}
\begin{frame}[fragile]
\frametitle{进程优先级}
\label{sec-3-1-6}
\begin{itemize}

\item 优先级(priority)与谦让值(nice)\\
\label{sec-3-1-6-1}%
top命令的输出中,PR是内核用于进程调度的动态优先级,而NI是用户可以设置的谦让值。NI可以影响PR,PR与NI越小则进程优先级越高。
nice值可以从-20至19,默认为0。
\end{itemize} % ends low level
\begin{exampleblock}{nice  以指定nice值启动进程}
\label{sec-3-1-6-2}


\begin{minted}[]{bash}
xeyes -center red &        #默认nice值为0
nice xeyes -center blue &  #默认设置nice值为10
nice -n 5 xeyes -center green &
ps -l   #查看进程的PR与NI值
\end{minted}
\end{exampleblock}
\begin{block}{注意}
\label{sec-3-1-6-3}

只有root用户可以在nice命令中为进程指定负nice值
\end{block}
\end{frame}
\begin{frame}[fragile]
\frametitle{进程优先级(2)}
\label{sec-3-1-7}
\begin{exampleblock}{renice  调整进程nice值}
\label{sec-3-1-7-1}


\begin{minted}[]{bash}
renice 15 12707     #将进程12707的nice值调整为15
renice -5 12675     #将进程12675的nice值调整为-5
renice 8 -u mike    #将用户mike的进程nice值调整为8
renice 0 -g users   #将用户组users的进程nice值调整为0
\end{minted}
\end{exampleblock}
\begin{block}{注意}
\label{sec-3-1-7-2}

只有root用户可以通过renice减小进程的nice值
\end{block}
\end{frame}
\begin{frame}
\frametitle{守护进程(Daemon)}
\label{sec-3-1-8}
\begin{itemize}

\item 守护进程(daemon)
\label{sec-3-1-8-1}%
\begin{itemize}

\item 守护进程是那些在后台运行的进程,脱离控制终端,守护进程与系统服务像关联,如日志守护进程syslogd、安全shell守护进程sshd等,守护进程名一般以d结尾。
\label{sec-3-1-8-1-1}%

\item 守护进程通常作为系统启动过程的一部分被启动,或者由root用户启动,有些守护进程以root用户身份运行,有些守护进程则以特定的系统用户身份运行。
\label{sec-3-1-8-1-2}%
\end{itemize} % ends low level
\end{itemize} % ends low level
\end{frame}
\begin{frame}[fragile]
\frametitle{at/batch 一次性计划任务}
\label{sec-3-1-9}
\begin{itemize}

\item atd守护进程:运行用户提交稍后运行的作业
\label{sec-3-1-9-1}%
\end{itemize} % ends low level
\begin{exampleblock}{用at命令提交作业}
\label{sec-3-1-9-2}


\begin{minted}[]{bash}
at 2:00 am
at> find / -mtime 0 -exec cp -p {} /bak \;
at> Ctrl-d
at 22:30 <task1
at -f task2 now+1h
atq           #查看作业队列,同at -l
atrm 2        #删除作业2,同at -d 2
man at
\end{minted}
\end{exampleblock}
\end{frame}
\begin{frame}
\frametitle{cron 周期性计划任务}
\label{sec-3-1-10}
\begin{itemize}

\item crond守护进程
\label{sec-3-1-10-1}%
\end{itemize} % ends low level
\end{frame}
\section{shell环境}
\label{sec-4}
\subsection{shell环境变量}
\label{sec-4-1}
\begin{frame}[fragile]
\frametitle{查看和修改shell环境变量}
\label{sec-4-1-1}
\begin{itemize}

\item 查看shell环境变量\\
\label{sec-4-1-1-1}%
\begin{minted}[]{bash}
env
echo $SHELL
\end{minted}

\item 修改shell环境变量\\
\label{sec-4-1-1-2}%
\begin{minted}[]{bash}
export HISTSIZE=500
\end{minted}
\end{itemize} % ends low level
\begin{block}{说明}
\label{sec-4-1-1-3}

上述修改仅对当前会话有效!
\end{block}
\end{frame}
\subsection{shell配置文件}
\label{sec-4-2}
\begin{frame}
\frametitle{bash配置文件}
\label{sec-4-2-1}
\begin{exampleblock}{系统级配置文件}
\label{sec-4-2-1-1}

\begin{enumerate}
\item /etc/profile       用户登录时运行
\item /etc/bashrc        启动bash时运行
\end{enumerate}
\end{exampleblock}
\begin{block}{用户级配置文件}
\label{sec-4-2-1-2}

\begin{enumerate}
\item .bash\_profile     用户登录时运行
\item .bashrc            启动bash时运行
\item .bash\_logout      用户注销时运行
\item .bash\_history     保存用户命令历史
\end{enumerate}
\end{block}
\end{frame}

\end{document}
